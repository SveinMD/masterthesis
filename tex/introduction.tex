\chapter*{Introduction}
\thispagestyle{chapterpage}
\addcontentsline{toc}{chapter}{Introduction}
\todoinline{Rewrite introduction}
Recent years has seen a large increase in the use of alternative energy sources, but petroleum products is still an integral part of the worlds energy supply. Diminishing production and new technological advances has made it possible to produce hydrocarbons from subsurface petroleum reservoirs previously thought to be non-profitable. 

In order to predict characteristics of the reservoir, the lifespan, production capacity, etc., a reservoir engineer will analyze field data and use theoretical models and experience to predict the future behavior of the fluids in the reservoir. Modern advances in computing power has allowed the development of full scale \emph{reservoir simulators}, technology that allows engineers to make educated decisions on control parameters during both the exploration phase and the production phase of a field. With the large investments in hardware it is obvious that fast and reliable simulation tools can prevent unsound investments and help maximize production and profit. 

For a realistic reservoir it is currently impractical to calculate flow patterns on the pore scale of the rock. The field data are instead averaged over grid blocks overlaid on the rock formations. The porosity, permeability, and other formation parameters are then assigned to each grid block, along with saturations and pressures of the fluid components in each grid block. This grid structure is used as a starting point for developing a discrete model of the reservoir over which \emph{conservation equations} are applied, leading to systems of \emph{partial differential equations}, often non-linear, governing transport of the different phases of the fluid in the rock formation. These equations are then discretized over the reservoir grid and implemented in a specialized computing package for simulation. The final important task is to visualize the data for human interpretation.

Although reservoir simulation has been an ongoing research area since the early seventies, it still has a host of challenging problems yet to be solved. The Open Porous Media Initiative (OPM) developed as collaboration project between a number of industrial players and research institutions, see \cite{opm_2014}, is an open source library seeking to supply researchers with a broad selection of efficient reservoir simulation tools in an accessible format.

The OPM library implements a range of numerical methods for solving the flow equations arising from the reservoir modeling. On such approach is to use a \emph{sequential splitting} scheme. The scheme splits the flow equations into an equation for the \emph{pressure} of the fluids and a separate set of equations for the \emph{saturation} of the different phases, which are the solved sequentially. The latter set of equations is often called the \emph{transport equations} since the new saturation fields of the fluids are found by solving these equations. The most straight forward way of solving the transport equation involves a large non-linear system of equations where all cell saturation are solved for simultaneously. It is possible to reorder this set of equations based on the flow field found in the pressure solver such that one can solve a series of single cell or smaller coupled problems, thus reducing the computational effort. The single cell problems are on the form ``find $x$ such that $f(x) = 0,~f\colon\mathbb{R}\to\mathbb{R}$'', that is, \emph{root finding problems}. Currently the OPM library solves the single cell problems using a modified version of the \emph{Regula Falsi method}. Several other root finding algorithms are known, both classic simple methods like the Newton-Raphson method, and more modern and involved methods like Brent's method. This project focuses on testing numerical methods for solving the single cell equations. A range of root finders are tested, among them Newton variants with update heuristics, like \emph{trust region methods}, which we will study two examples of, namely the trust region method due to \citet{jenny_unconditionally_2009}, and the the more recent method due to \citet{wang_trust-region_2013}. These and other methods will be presented, implemented in the \opm framework and tested against the current solver.

Chapter \ref{chapter:porous_media_flow} presents a short introduction to petroleum reservoirs before developing the conservation equation and the sequential splitting scheme. Next, Chapter \ref{chapter:numerical_methods} start by presenting the sequential splitting scheme and the finite volume discretization of the flow equations. The reordering approach is also presented before the root finding algorithms used to solve the residual equations are discussed. Finally, Chapter \ref{chapter:numerical_results} presents numerical results comparing the new methods with each other and the existing methods in the \opm library.
\newpage