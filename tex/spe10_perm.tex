\begin{figure}[!ht]%
\centering%
\tikzsetnextfilename{spe10_perm_tarbert}
\begin{subfigure}[b]{0.49\textwidth}%
\centering%
\begin{tikzpicture}%
    \node[anchor=south west,inner sep=0] (image) at (0,0) {\includegraphics[width=\textwidth]{figures/spe10_tarbert_view.eps}};%
\end{tikzpicture}%
\caption{Top view}%
\label{fig:tarbert}%
\end{subfigure}%
\centering%
\tikzsetnextfilename{spe10_perm_upperness}
\begin{subfigure}[b]{0.49\textwidth}%
\centering%
\begin{tikzpicture}%
    \node[anchor=south west,inner sep=0] (image) at (0,0) {\includegraphics[width=\textwidth]{figures/spe10_upperness_view.eps}};%
\end{tikzpicture}%
\caption{Bottom view}%
\label{fig:upper_ness}%
\end{subfigure}%
\caption{Inhomogeneous permeability data from the second SPE10 data set \citep{spe10_2000}. The top 35 layers are part of the Tarbert formation. The lower 50 are part of the Upper Ness formation. The model dimension is  $\unit[1200]{ft} \times \unit[2200]{ft} \times \unit[170]{ft}$ with $60\times220\times85$ cells.}%
% , approximately $\unit[335]{m} \times \unit[670]{m} \times \unit[52]{m}$,
\label{fig:spe10_perm}%
\end{figure}%